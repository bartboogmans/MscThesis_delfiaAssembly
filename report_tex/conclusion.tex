\chapter{Conclusion}
\label{Conclusion}
This paper presents systematic development steps and design choices to realize a multivessel system able to configure vessels into interconnected platforms while performing perform configuration dependent collaborative control. 

The developed framework was designed to reflect a scenario of an automated fleet system with the task to solve a logistical problem, by forming floating infrastructure in a predefined shape that can be loaded, which subsequantially needed to perform motion tasks. Implementation serves as proof of concept, which is evaluated as performing compliant to the set design criteria. Full scale commercial adoption will bring more challenges that were not present in the experimental lab setup, such as disturbances due to wind and current.  Used design approach is considered applicable to other scales and environments as all system components are considered replacable for other solutions better serving requirements of the usecase.

The research approach relied on searching answers to the following main research question ans subquestions:

\noindent {\large\textbf{ Main research question}}
\begin{itemize}
	\item How can a fleet of modular surface platform vessels be controlled to achieve automated assembly and configuration-dependent platform control?
\end{itemize}
\noindent {\large\textbf{Sub-questions}}
\begin{enumerate}
	\item What is the state of the art within automated vessel platforming systems?
	\item What characteristics does a vessel system have that integrates automated assembly with configuration dependent control?
	\item How can the dynamics of the multi-vessel system be represented?
	\item How can a fleet control framework be developed that performs automated platform assembly and configuration adaptive platform control? 
	\item What is the performance of the developed system?
\end{enumerate}

Literature survey describes major literature of projects related to vessel self-assembly and collaborative control approaches to answer the first subquestion. The review describes various works categorized as 'vessel platform self-assembly' and one project showing development of a 'collaborative and coordinated multi-vessel-platform control'. It seemed natural that modular vessel platform automation will encompass behavior of both categories in the near future to become of highest societal benefit. Yet no sources were found describing both behaviors integrated in a single system, leading to the gap of knowledge that this project adresses. 

A more in depth look is given to the definition of multi-robot systems that perform 'self-assembly' and 'collaborative, coordinated control' to answer the second subquestion. This section significantly relied on literature works to map characteristics of both systems from different perspectives. Works with a general robotics perspective (versus marine robotics) were shown to have significantly more extensive descriptions of traits, design choices and categorizations, and thus were included within characterization efforts.  It was shown that many approaches were possible to design the discussed systems, each with their own benefits and limitations with different probable usefulness for logistic purposes. Generalized system characteristics are distinct and narrowed down set of solutions that can be considered more feasible to perform in commercial applications, taking into account the design trend in other projects and their emergent characteristics.

A multi-vessel state description is proposed, together with a novel approach to predict dynamics of a combined waterborne structures to answer the third subquestion. Existing (single) vessel state descriptions are extended to formulate this multi-vessel state description aiming to enable consistent system notation of scenarios pertaining more than a single object (vessels, coordinate systems) that can be referred to. Furthermore, the concept and definition of a platform-coordinate system is introduced. This resulted in three distinct types of coordinate systems; vessels, assemblies/platforms and global (e.g. inertial) frames that define state of an object and coordinate system in which parameters of other objects (pose, velocities, forces) can be expressed. 
The proposed approach to predict dynamics of a combined waterborne structure utilizes superpositioning of module models into a combined platform model as alternative of parameter estimation experiments for variable configurations. Particular emphasis is given to taking into account platform shape (besides numbers) to formulate a combined model that inherits directional dependent terms, such as hydrodynamic added mass or directional dependent dampening. Connections between modules were assumed rigid to allow expression of module modules in a single point and orientation (such as the platform frame) analytical by means of coordinate system translation and rotation to formulate the combined expression. A key requirement of this approach is that individual module models sufficiently describe the behavior of a model also in very close proximity of other modules, or discrepancies due to module proximity are sufficiently negated by additional compensating terms. The proposed platform model can form a building block for approximated dynamics of the combined structure utilizing obtainable parameters (individual module models), which can be extended to include vessel proximity effects if needed. 

An experimental fleet control system incorporating assembly and collaborative control automation has been developed to answer the fourth subquestion. A homogeneous set of Delfia-1* model scale robotic vessels formed the fleet of modules for the proof of concept in indoor lab environments. Major design decisions are as follows:
\begin{itemize}
	\item The multi-robot network topology and hierarchy is centralized on a platform level, to support collaboration and a framework for coordination. This means that network topology changes according to changing configurations. 
	\item The overall platform-level control approach is subdivided into modular subsystems. The tasks of these subsystems can be solved in various ways, of which the chosen solutions are explained and design choices motivated. 
	\item The approach of controlling a multi vessel platform in a collaborative manner uses the proposed approximate platform model dependent on configuration size and shape. The platform model is formed by combining models of connected modules, which were assumed known, taking into account individual module dynamics, placement, and orientation. 
	\item Desired control effort is generated with a set of parralel PID controllers that adapt to estimated platform parameters, such as configuration dependent maximum thrust and system inertia. 
\end{itemize} 
Chapter \ref{chap:evaluation} answers the fifth and last subquestion by evaluating the performance of the final implemented system, starting with the ability of self-assembly, following with performance of the motion control system and it's ability to operate on any configuration. 

It is concluded that the system developed througout this project is compliant with the set design criteria. Self assembling behavior was evaluated by expressing relative module motion as key performance indicators, which should be ideally zero if connected. This behavior was indeed observed, where the maximum motion after connecting in a 10 second period was within bounds of $1.0408e-3$ and $0.8995e-3$ meter for x and y motion and within $6.738e-3$ radians ($\approx 0.3860$ degrees) for rotation. 


Main objectives of the motion control system were met, as the considered configurations converged to changing reference inputs in a reasonable time without reliance on configuration-specific methods. Performance of the fleet motion control system is evaluated from reference step responses on all considered degrees of motion (forward, sideways and rotating) of single vessel and assembled 3x1 platform configuration.  The configuration dependent control system was aimed to effectively utilize available actuators while remaining in control of a changing dynamical system using the approximate platform model. The control to a single entity allowed coordination strategies to increase actuator usage effectiveness. Furthermore the controller should show acceptable adaptation to varying configurations. 

As the control approach is intended to work on arbitrary platform configuration, it is evaluated for two configurations, namely single operaration and as a 3x1 lattice structure.  Performance of the fleet motion control system is evaluated from reference step responses on all considered degrees of motion (forward, sideways and rotating) by expressing rise time, settling-time and overshoot as key performance indicators (with step amplitude as characteristic motion of 0.5m \& 1m for translation and a 90\textdegree rotation). Behavior in terms of overshoot was compliant with the design criteria (<50\%) for motion of both configurations along all dimentions averaging 34,7\% for single and 35,8\% for 3x1 configured assembly. Design goals  for settlingtime (response within factor 0.05 of steady-state in $t_s<40s$) were met through all single operation and in 83.3\% of configured tests. Risetime behavior proved significantly within design criteria ($t_r<10s$) in all directions and configurations, typically under four seconds. 

The discussed criteria on behaviors of self-reconfiguration and collaborative control simultaneously within the same framework are considered achieved. It is concluded that a modular vessel platforming system can be equipped with features automating reconfiguration and collaborative and coordinated control, where the following major challenges are identified as follows
\begin{itemize}
	\item Both individual behaviors can already be implemented in a wide variety of approaches, where integration of the two widens the design spectrum significantly more. Both behaviors can be considered complex by themselves, similarly increasing in a combined framework. This complexity toughens challenges of design, but also in facets realizing robustness of a multi-layered system.
	\item Earlier developed works showed iterated views on solutions to achieving the desired behaviors. It was found that a main factor for succesful integration is ensuring that both behaviors are interoperable. This particularly means that platform motion control systems need to be able to adapt in real time to configuration changes. 
\end{itemize}

\section{Discussion}
%% ----------  Non optimality of subsystems 

All solutions for subsystems aimed to let the system perform in a lab setting with a fleet of Delfia-1* modules, yet it must be said that these design choices are not optimized or guaranteed to perform on use cases with a different fleet, scale, environment and goals. Replacing or improving components of the system is stimulated by making the control framework as modular as possible. Various subsystems can be further developed, which was expected as exploration of a novel combination of behaviors was the goal rather than optimization. That said, the reader is invited to use the approach and description of implementation for inspiration to develop waterborne multi-vessel collaborative systems.


%% ---------  Appropriateness of the configuration adaptiveness
The control approach uses the concept of an approximate platform-model to then influence motion control behavior similar to \citet{park2019coordinated}, but the formation of this appropriate model differs. Their estimates of rigid body inertia scales by $n$ and $n^2$ for translation (mass) and rotation (moment of inertia) respectively, yet this approach does not take platform shape into account. Their scaling rules make sense in some respects, but are simplistic in others. %Estimated mass scaling by the number of connected modules $n$ seems sensible and the quadratic scaling of moment of inertia would hold if the shape of the scaled object (streched equally in along the water surface directions) would not change. 

There are two factors that are considered to limit estimated model accuracy. Firstly, the actual moment of inertia of a combined structure will differ depending on configuration shape, and not solely on numbers. Units at distance from the centre of mass provide far greater contributions to inertia than centered ones, similar as to how the moment of inertia of a flywheel and an equally weighted solid rod greatly differ, explained by the parralel axis theorem.
Secondly, a vessel's hydrodynamic effects are directionally dependent, thus orientation and placement of a module with respect to the rest of the platform will affect hydrodynamic forces acting on it, thus taking this into account has the potential to find better platform representations. 

Various models for hydrodynamic forces on a ship exist, often distincting accelleration and velocity dependent terms being modelled as "added mass" and dampening. Accelleration dependent contributions of hydrodynamic forces are commonly modelled as constant but not  equal in all directions, yielding satisfactory accuracies in most conventional ship use cases \citet{humphreys1978prediction}. Is questionable to what extent such constant added mass terms remain an accurate model in the scenario of vessel assembly where modules operate in unconventionally close proximity. Yet given that (at least some part of) the hydrodynamic forces on a module are accurately represented by this constant directional dependent added mass, then the orientation and placement of that module should be taken into account when estimating contribution to the assembled structure. 

Various concepts that are discussed in this thesis have the potential to be competitive problem solvers, yet feasibility needs to be always taken into account.
Increased combined platform model accuracy (of a complex model estimator) is naturally desirable, yet one should ask what requirements of model estimators are. Perhaps a simple, less accurate (but satisfactory) approach can have outweighing benefits in terms of speed and cost effectiveness. Alternatives of predicting a dynamical platform model also exist. Parameter identification experiments can be conducted for all reasonably expectable configurations if they are not too numerous, or a controller can be designed to quickly learn from its responses to formulate a model once it encounters an unfamilliar configuration. 
The concept of reconfigurable robot systems may not be competitive, or only for a very small niche of logistic challenges. \citet{seo2019modular} notices that of its three main benefits of MMR systems, Versatility, robustness and low cost, only Versatility seems to commonly improve while robustness and cost are usually decreasing for revieuwed literature showing research prototypes. As development of MMR systems is still an active topic of research, we can hope to see this leading to MMR systems becoming more viable in the future. Positive system aspects can be further enhanced, while negative aspects are reduced or reduced to managable levels. This hopefully leads to a scenario where positive aspects far outweigh the negative, such that reconfigurable vessel systems become a commonly applied powerful logistical tool from a social, technical and economical vieuwpoint. 


%% ---------  Representativeness of the 2 configurations

Throughout experiments only two configurations were tested, although the designed control system is aimed to function on any arbitrary configured platform. The results on the two configurations are positive, and indicate good hopes for performance on many other configurations, although this needs to be validated. The described approach on adapting to unknown configurations bears foundations which can be further explored to improve potential control over arbitrary configured dynamically shapeshifting vessel platforms. 
%The designed framework theoretically supports motion control of any arbitrary configuration, as was desired. Experiments were conducted on two quite general shapes that both showing satisfactory responses. The control approach adapts to unknown configurations by using scaling rules aimed to be representative over a wide variety of shapes and sizes. Although scaling rules have been designed relying on physics, used assumptions need to be re-evaluated for correctness of future configurations, particularly with more 


\section{Recommendations}
\subsection*{Further investigation of arbitrary configured control approaches}
The amount of works presenting approaches of controlling arbitrary configured modular marine structures is limited, as only \citet{park2019coordinated} was found adressing that particular challenge. Many design choices from the system developed in this work are similar to this earlier work, such as adoption of platform-level centralized control topology. Distinction between control effort generation and allocation was already common although this work applied it to a coordinated multi-robot structure. \citet{park2019coordinated} also presented the concept of using an approximate platform model estimator, which can be a useful tool supporting platform control performance if the estimation proves of sufficiently accurate. Various facets of this multi-robot collaborative system can be further investigated, as already initiated in section \ref{analysisConfigAdaptation} of this work by searching for fundamental system characteristics and common approaches described more broad in general robotic literature. 

Collaboration and coordination in a multi-robot structure is aims to increase some facet of the system's performance to increase overall system performance over the sum of individuals. Decentralized topologies could also support such behaviors while reducing single points of failure, although centralization has other benefits. 

Other approaches into formulating an approximate platform model might prove more accurate. This work presents a novel platform model based on a combination of module models that make up the body in section \ref{platformModel} and appendix \ref{appendix:CombineDynamics}. It assumes that the individual module models are to some extent representative operating in proximity of surrounding modules in a configuration, or that compensating factors can be formulated, which are both bold assumptions. Hydrodynamic effects on arbitrary shaped platforms will not likely be estimated perfectly anytime soon, but that is also not the goal. However, formulating rules of thumb or approximations that perform to a great extent to reasonably expectable scenarios would be of great benefit. For example, the concept of hydrodynamic added mass from  \citet{humphreys1978prediction} is a rather rough simplification of reality, yet it is commonly used in marine control technology as it often provides a great trade-off between simplicity and representativeness. The platform model described in this paper conserves hydrodynamic effects of a module model that show directional dependence, such as hydrodynamic added mass, or dampening models depending on direction of motion. 
It is important to ask to what extent such models need to be accurate rather than a rough estimate. For future research it is thus suggested to explore benefits of platform model accuracy in use cases of modular marine robotic systems with varying objectives, scenarios and scales. Governing factors affecting vessel platform dynamics can be further investigated. Existing models can be assessed, improved, or new ones can be formed. 

\subsection*{A framework of standardization, modularity, interoperability and solution sharing.}
There are many problems that have been encountered during the design of the system's the proof of concept in this paper of which engineers will face similar problems in the future, and can waste effort finding suitable solutions for their problems that others already faced. A framework that allows sharing and integration of solutions to challenges will be key in global adoption of automated vessel systems. This would reduce constant re-invention of the wheel would instead allow developers to effectively add development upon existing. ROS is  as an impactful standardizer and facilitator of robotic middleware adopted globally. For automated marine systems, an accepted sharing platform for solutions would greatly benefit development, standardization, module reusability interoperability and thus accellerate adoption of marine automation to benefit society. It is recommended to focus on development and adoption of such a shared platform in successive research. This research could investigate if this is best founded from existing middleware, such as a ROS-maritime-control branch, built upon other frameworks or stand-alone. Key functions and tasks can be further defined and categorized as modules and subfunctions reflecting general marine-control science and industry. Different stakeholders could effectively develop interoperable technical solutions with varying incentives. Sales could provide financial gains from developing competitive solutions to subsystems under some licenses, while contributions under an open source licence can create a snowball effect in developments. 

\subsection*{Support creation of shared consistent definitions}
To avoid misinterpretation as introduced in section \ref{literatureDefinitions}, a clearly defined definition of autonomy in the context of shipping needs to be developed. The International Maritime Organisation is undertaking this. Yet now the interpretation of IMO's proposal is very different than how it is used in other sciences than engineering, which worries me. A strong definition of autonomy that is conceptually different than automation has the possibility to be a very useful term in development of marine technology. These standards developed by IMO are estimated to become leading definitions once the iterative changes come to an end. It is recommended that anyone using related terms keeps an eye on developments of terminology standardization of important organisations such as the IMO, and strive to contribute formulation if able. 

\subsection*{Keep considering what to automate}
The fact that we can automate something, doesn't mean we should. It is not by definition desirable to make vessels machine controlled. Some times troughout writing this paper the notion was encountered that full vessel system automation is the ultimate solution. However, humans can be great controllers of vessels, generally seen as versatile and having great ability of adaptation to a situation or environment. Machines can also do certain tasks better than humans, where raw calculation jobs can be seen as an extreme example. As control technology advances, division of tasks between humans and machines can slowly shift, as machines hopefully continue to become better at doing some tasks, allowing humans to use their time for other things. Judgement of performance is off course not to be simply based on quality of vessel operation (who can steer the vessel the best?), but also on various other aspects such as financial, social or safety. It is important to keep asking ourselves: "How would we benefit from automation in this situation?", and not automate for the sake of automation. Let humans do what they do good, and likewise for machines.
