\chapter{Executive Summary}
%\addcontentsline{toc}{section}{Executive Summary}
Shipping enterprises are under constant pressure to perform faster and more flexible transportation with lower costs and emissions.  This incentive, together with advancements such as in communication and network technologies, have led to various projects that explore vessel control automation in the marine sector as a solution. 

In the past decades, many projects worked to further explore automating various parts of vessel control. Further automation of ship control tasks can broaden viable applicable scenarios, reduce the need of human operators, increase transport system effectivity, reliability and safety, while consuming less resources. Realistically, automation will rarely improve all the named benefits at once. Projects tend to aim to enhance several of these measures of performance that are most relevant for a use case, while allowing acceptable losses elsewhere. Automation, collaboration, negotiation and information sharing allow a wider variety of solutions to logistic challenges.

% Introduceer trend in platforming: What is it? So why is platforming relevant? 
A recent concept is utilization of automated modular waterborne structures to perform tasks such as forming temporary infrastructure or create arbitrary shaped vessels. Such structures on the water surface form vessel platforms, built from a number of smaller connected vessels, and have  been an area of research. Such vessel platform systems could in some cases be applied  for quick, flexible and affordable solutions to logistic challenges where no other vessel solution is feasible for that task, environment or timeframe. 

The main objective of this research is to develop a control framework for a modular fleet capable of assembly and collaborative motion control of connected structures. An approach on combining two systems (automated assembly \& collaborative control) is presented throughout this paper supplemented with various insightsto aid further developments. First steps are identification of characteristics of the integrated systems, followed by proposing a representation of multi vessel state notation and platform dynamics, which are subsequentially all used to aid design decisions in developing an experimental setup. Finally the performance of this system is evaluated trough experiments. 

\section*{Literature Survey}
For effective discussion on ship automation good definitions of terminology are required to avoid misinterpretation. Terminology used in this work is thoroughly discussed. Used definitions are stated, supplemented with various other common interpretations. It is shown how interpretation of concepts such as 'autonomy', 'smart-ship' and 'automated' leave much room for discussion, with dangers of misaligned expectations between discussion partners. Ongoing efforts aim to standardize such terminology.

Literature shows various works on automated control of vessel platforming systems, yet it was noticed that the majority of relevant works focus on either the "assembly process" or "collaborative motion control", so they are discussed in those groups. 

Two projects have been identified to build up the majority of the literature on automation of fleet assembly into floating structures. The first discussed works are on a project that developed a nameless fleet of container-like modules with rope and hook connectors, while the second works are developed in the "Roboat" project with vessels of similar naming. Both projects worked on model scale and focussed on developing various facets of assembly in multiple papers. Challenges regarding multi-vessel-structure reconfiguration are adressed from a hardware (e.g. connector between ships), latching strategy and task scheduling perspective.  A wider variety of approaches to realize automated multi-robot reconfiguration can be found in general robotic science compared to marine systems.

%% Collaborative motion control
As arbitrary shaped waterborne structures are formed, it can be beneficial to be able to manipulate motion of the combined body with collaborative approaches. Works describing motion control of modular systems are limited, as the majority of the encountered literature focusses on controlling motion of a single large object  (e.g. a large unactuated barge) by utilizing combined effort of  a fleet of automated vessels (e.g. tugboats). Collaborative control approaches of a single (non-modular) object by multiple entities are discussed as these works present insights and approaches to aid effective design choices. One publication proposes an approach to control a modular structure, utilizing PD control for control effort generation, control effort allocation with energy optimizition and an approximate-platform-model to scale control parameters to variable configurations. 

\section*{System Characterization}
To look beyond the design solutions from existing projects and create a broader view on the design spectrum, a deeper look is given into the meaning of automated 'reconfiguration' and 'collaborative motion control' of modular vessel structures. Generalized system characteristics are distinct and narrowed down set of solutions that can be considered more feasible to perform in commercial applications, partially based on design choices of other projects.  Automated modular vessel platform reconfiguration systems can be categorized as a more general "modular reconfigurable robot" system in a maritime setting. The key characteristic of such systems is described as adaptability of hardware structure to suit a given task or environment. MRR systems differentiate themselves from normal robot systems by determining and executing a course of action, to change it's configuration. Various requirements of floating MRR systems were identified during the course of the project. Some approaches originating from sources about robotics in general were broader than approaches encountered in sources about modular vessel platforming. During elaboration on system characteristics various projects are discussed, more specifically, how they solved their challenges and in which characteristics did this result in.

As various characteristics of automated 'reconfiguration' and 'collaborative motion control' systems are shown, more attention is given to integration of both systems. An additional characteristic is identified as these two behaviors are integrated in one framework. As a platform configuration changes through time, the motion control system needs to be able to support real time varying configuration parameter, such as estimated dynamics, size, shape, or centre of mass, or in some cases even the network topology. Various considerations for multi robot systems have been observed such as control structure topology (e.g. centralized, decentralized, hybrid), Homogeneouity of the fleet (similar modules or varying), actuation (e.g. choice and layout of connectors, thrusters) and utilization of network (task distribution, decentralization).

\section*{Multi-vessel system representation}
Existing vessel state descriptions are used to formulate a multi-vessel state description. It aims to enable consistent system notation of scenarios pertaining more than a single vessel, as mult-vessel frameworks have more objects that can be referred to. Furthermore, the concept and definition of a platform-coordinate system is  introduced. This resulted in three distinct types of coordinate systems; vessels, assemblies/platforms and global (e.g. inertial) frames that define state of an object and coordinate system in which parameters of other objects (pose, velocities, forces) can be expressed. 

For collaborative motion control of modular objects an approximate dynamical model is often desired. As model parameter estimation experiments are often infeasible for every configuration of a combined structure, a prediction of dynamical behavior is proposed using the often known dynamics of modules. Assuming rigid connections between modules allow expression of module modules in a single point and orientation (such as the platform frame) analytical by means of coordinate system translation and rotation. A key requirement is that individual module models sufficiently describe the behavior of a model also in very close proximity of other modules, or discrepancies due to module proximity are sufficiently negated by additional compensating terms. Module model terms that are directionally dependent, such as hydrodynamic added mass, remain represented in the proposed platform module, taking into account the modules placement and orientation. The proposed platform model can form a building block for approximated dynamics of the combined structure utilizing obtainable parameters (individual module models), which can be extended to include vessel proximity effects if needed. 

Both the multi-vessl state notation and the proposed platform model will be used in the next chapter during the development of a control system of the described structure. 

\section*{Development of an automated assembling \& collaborative motion control system}
A multi-vessel platforming system is designed and implemented. Major considerations and characteristics of automated reconfiguration and collaborative motion control systems are used to support design choices to meet the following set criteria; The system is required to perform automated vessel platform reconfiguration, while simultaneously showing collaborative platform motion control behavior. The framework should support a large amount of-, or arbitrary  configurations.  This creates adaptiveness to a wide set of tasks for modular vessel platforms, which is a key element supporting commercial competitivity. Furthermore, developed solutions are aimed to be general, such that they are applicable or at least meaningful on other ship-systems, environments and scales of operation. This is considered important to stimulate that knowledge gained from the developed experimental setup benefits future commercial implementation. Finally, solutions are aimed to be modular, such that subsystems are designed to be conveniently swapped out for another (perhaps improved and better performing) version, easing future improvements and increasing reusability of work.

%% Key performance indicators! XXX I WAS HERE

The goal of this design is not to optimize an existing system, but to explore a novel combination of behaviors that is expected to be of interest in the near future. 
The ability of succesful assembly needs to be proved, yet successrate and reconfiguration-speed need only be in reasonable limits and magnitude to reflect commercial implementation. Cooperative motion control needs to show convergence of the system to a desired state while a network of robots operate more effective than the sum of individuals, yet quantitiative motion responses need only be stable and within reasonable timeframe. 

Content on design of the experimental setup starts from a high level view by discussing the multi-robot network topology adapting to varying configurations. It continues to explain the general control approach and division in subsystems that each solve a part of the control problem. Design choices of each subsystem are subsequently discussed in terms of model estimation, state estimation, control effort generation, control effort allocation and  assembly protocol. 

The modules used are Delfia-1* robotic model scale, homogeneous, rectangular vessels equipped with two rotating azimuth thrusters. The design has two axes of symmetry, including weight distribution and thruster placement.

The approach of dividing a vessel control system via the Guidance, Navigation and Control categorization is used to distinguish between different system processes, yet the control for this particular system does refer to that of a single vessel, but rather to a set of vessels, which sometimes are controlled individually (when a module moves alone), and in other times together (in assembled platform). Hence, due to shapeshifting, not only the system behavior changes (such as inertia of a platform of variable shape and size), but also the control structure topology.
A layered control structure has been developed, operating on three levels; fleet (guidance), platform (control) \& module (actuators \& physical layer). The fleet control layer represents a guidance system, responsible for high level task planning and providing motion control objectives. These objectives are realized by the platform-motion-control layer in centralized fashion, where each connected structure has control decisions made by this single entity. Generated actuator commands are then sent to all modules of a platform to be executed. 

The guidance layer is implemented rather simplistic as setpoint regulation, where assembly and motion control objectives are provided through scheduling and operator input to provides sufficiently varying tasks for system behavior evaluation. Platform motion control generates control effort with three parralel proportional integral \& derivative (PID) controllers that scale output on approximate platform model parameters to remain in control of arbitrary configured structures. The platform motion controlling agent allocates the desired virtual control effort between modules such that is realized as actual resultant forces and torque on the structure. 

The experimental setup is designed to operate in the towing tank facility of section Maritime and Transport Technology (section MTT) in the faculty of Mechanical, Maritime and Materials Engineering (Faculty 3ME). One of such tanks is equipped with an optical sensing and interpretation system from the brand Optitrack to provide module state estimation. Communication between entities is facilitated by through topicwise subscriber-publishing protocol of the Robotic Operating System (ROS).

\section*{Experimental evaluation}
Developed system performance evaluation is done in two steps, namely in terms of assembly and motion control behavior. 

Performance of reconfiguration is quantified by evaluating the change in relative pose between neighbouring modules in all considered degrees of freedom. Measured signals showing relative module positions show plateauing behavior indicating hull contact and succesful assembly if it occurs in all degrees of motion. Such sudden stops of motion in all dimentions were found and it was shown how succesful assembly can be found numerically.

Motion control performance is evaluated for two platform configurations; a single vessel scenario and a 3x1 assembled platform. Results of system behavior are presented and performance is quantified. Tests consisted of step inputs in the reference signal on all degrees of motion independently. All motions were analyzed by expressing risetime, settlingtime and overshoot as key performance indicators and compared with design criteria.  It is concluded that the proposed multi-vessel control system is able to perform automated reconfiguration and collaborative motion control, as behavior proved in line with design goals. 

The developed framework has certain limitations in it's current form of implementation that require further development, both in terms of research and commercial implementation. Yet the overall emergent behavior of the automated multi-vessel system is positively received as a small step towards improving mankind's logistical tools. 